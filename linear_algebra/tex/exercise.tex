% !TeX spellcheck = de_DE
\documentclass[a4paper, 12pt]{article}

\usepackage{bm, bbm, wrapfig, float, setspace, amsmath, amssymb, amsthm, url, graphicx, ngerman, transparent, enumerate, bbold, esint, polynom, hyperref, microtype, etoolbox, braket, cleveref, hyphenat}
\usepackage{centernot}

\usepackage{tikz}
\usepackage{epsdice}
\usepackage{listings}
\lstset{language=Python}
\tikzset{>=stealth}
\usetikzlibrary{decorations.markings, shapes, calc}

\renewcommand{\familydefault}{cmss}   % Generates sans serif fonts


\renewcommand{\l}{\left(}
\renewcommand{\r}{\right)}
\newcommand{\gs}{\text{gs}}
\renewcommand{\P}{\hat{P}}
\newcommand{\U}{\hat{U}}

% \newcommand{\bra}[1]{\langle#1|}
% \newcommand{\ket}[1]{|#1\rangle}
\newcommand{\bkt}[2]{\left\langle #1 |#2 \right\rangle}
\renewcommand{\ij}{{\langle \vec{i}, \vec{j} \rangle}}
\renewcommand{\H}{\hat{\mathcal{H}}}
\newcommand{\Ht}{\tilde{\mathcal{H}}}
\renewcommand{\c}{\hat{c}}
\renewcommand{\a}{\hat{a}}
\newcommand{\cd}{\hat{c}^\dagger}
\newcommand{\rh}{\hat{\rho}}
\newcommand{\rht}{\tilde{\rho}}
\newcommand{\ad}{\hat{a}^\dagger}
\newcommand{\bd}{\hat{b}^\dagger}
\newcommand{\ubd}{\hat{\uline{b}}^\dagger}
\newcommand{\ub}{\hat{\uline{b}}}
\renewcommand{\b}{\hat{b}}
\newcommand{\hd}{\hat{h}^\dagger}
\newcommand{\h}{\hat{h}}
\renewcommand{\d}{\hat{d}}
\newcommand{\n}{\hat{n}}
\newcommand{\D}{\hat{D}}
\newcommand{\Dd}{\hat{D}^\dagger~\hspace{-0.12cm}}

\newcommand{\G}{\hat{\Gamma}}
\newcommand{\Gd}{\hat{\Gamma}^\dagger}
\newcommand{\F}{\hat{F}}
\newcommand{\Fd}{\hat{F}^\dagger}
\newcommand{\hc}{\text{h.c.}}
\newcommand{\MF}{\text{MF}}
\newcommand{\BEC}{\text{BEC}}
\newcommand{\RG}{\text{RG}}
\newcommand{\psd}{\hat{\psi}^\dagger}
\newcommand{\ps}{\hat{\psi}}
\newcommand{\I}{\text{I}}
\newcommand{\p}{\text{p}}
\newcommand{\f}{\text{F}}
\newcommand{\s}{\text{S}}
\renewcommand{\sf}{\text{MIX}}
\renewcommand{\O}{\hat{\mathcal{O}}}
\newcommand{\W}{\hat{W}}
\newcommand{\Ud}{\hat{U}^\dagger}
\newcommand{\HMF}{\mathscr{H}_{\text{MF}}}
\newcommand{\ph}{\text{ph}}
\newcommand{\IB}{\text{IB}}
\newcommand{\B}{\text{B}}
\newcommand{\eff}{\text{eff}}
\newcommand{\tr}{\text{tr}}
\newcommand{\tdiff}{\,\mathrm{d}}



\normalsize
\setlength{\hoffset}{-1.5cm}
\addtolength{\textwidth}{3cm}
\setlength{\voffset}{-2.2cm}
\addtolength{\textheight}{3.5cm}
\addtolength{\footskip}{0.2cm}

%\input{macros.tex}
\newtheorem{aufgabe}{Problem}
\newtheorem{bspaufgabe}{Beispielaufgabe}

\usepackage{titlesec}

\titleformat{\section}[block]
{\normalfont}{\textbf{Problem \thesection}}{1em}{}

\DeclareMathOperator{\dd}{d}
\newcommand{\eul}{\text{e}}

% imaginary unit 
\newcommand{\imag}{\textsl{i}}
%
% with or without solution?
%
\makeatletter
\@ifundefined{solfalse}{
\newif\ifsol
\soltrue
}
\makeatother
\newcommand{\solA}{\vskip0.5em \color{brown}\noindent\textbf{Solution:\\}}
\newcommand{\sol}[1]{\vskip0.5em \color{brown}\noindent\textbf{Solution:}\\#1}

\usepackage{relsize}
\newcommand\Cpp{C\nolinebreak[4]\hspace{-.05em}\raisebox{.4ex}{\relsize{-3}{\textbf{++}}}}

\begin{document}
	   
% ********** First Box **************
\vspace*{-13mm}
\noindent\fbox{
	\parbox[b][19mm][b]{16mm}{%\parbox[POS][H�HE][POS-INNEN]{BREITE}{INHALT}
		\center\includegraphics[trim=1.6mm  0 0 0, width=18mm]{../../images/lmu-bildmarke.png}
	}
}
\hfill
% ********** Second Box **************
% Diese Schrift sollte so gross sein, dass die Oberkante des Ms von
% Max mit der Oberkante von LMU abschliesst
\fbox{\parbox[b][19mm][b]{19mm}{\includegraphics[width=19mm,height=19mm]{../../images/lmu-wortmarke.png}}}%
\hfill\hfill
% ********** Third Box **************
\fbox{\parbox[b][19mm][b]{84mm}{\fontsize{9}{12}\sc Faculty of Physics, Summer Term 2023 \\  Numerical Quantum Physics\\ 
		Lecturer: Dr. S. Paeckel \\ Assistant Lecturer: Z. Xie, B. Schneider
		\vspace{0.1mm}}}
\hfill
% ********** Fourth Box **************
\fbox{
	\parbox[b][19mm][b]{25mm}{\hspace*{3mm}\transparent{0.52}{\vspace{-1mm}\includegraphics[trim=0 6mm 16mm 20mm, clip, height=21mm]{../../images/lmu-siegel.png}}\transparent{1}}
}
\begin{center}
	\small \url{https://www2.physik.uni-muenchen.de/lehre/vorlesungen/sose_23/nqp/}
\end{center}
%
\vspace{8mm}
%
\centerline{\Large\textbf{Sheet~1:~Linear Algebra}}
%
\vspace{3mm}
%
\normalsize\centerline{Released:~04/21/23;~Submit until:~05/05/23 (\textbf{20 Points})}
%
%\vspace{0mm}
%
%
\vspace{6mm}
%
This sheet is about a very small introduction into elementary concepts of \Cpp{} and how to use them to implement fast linear algebra operations using Intel's \underline{M}ath \underline{K}ernel \underline{L}ibrary (MKL).
%
In order to work through the problems, checkout the \texttt{git} repository \href{git@gitlab.physik.uni-muenchen.de:nqp/nqp-exercises.git}{nqp-exercises} provided on the lecture's \href{https://www2.physik.uni-muenchen.de/lehre/vorlesungen/sose_23/nqp/uebungen/index.html}{homepage} and work with the code templates.
%
While the header file \texttt{la\_wrapper.h}, which contains an interface to some linear algebra methods of the MKL, is complete, in each code template there are files which are erroneous and need to be corrected.
%
In a \href{https://jupyter.physik.uni-muenchen.de}{Jupyter-Hub} session, use the provided \textit{Makefile} as well as a proper make.inc to compile the corrected code templates.
%

%
\section{Implementing a matrix class \textbf{(5 Points)}}
%
Checkout the code template \texttt{blas\_wrapper}, which contains an elementary implementation of a wrapper of the widely used linear algebra MKL.
%
\begin{itemize}
	\item[(1.a)] \textbf{(3P)}
	%
	Correct the code template \texttt{blas\_wrapper} such that you can compile it using \texttt{make build}.
	%
	As a consistency check, you can use the default implementation of the \texttt{main}-routine and check the output when executing the compiled binary.
	%
	\item[(1.b)] \textbf{(2P)}
	%
	You now have a rudimentary matrix class that implements fast linear algebra operations using Intel's MKL.
	%
	Use this class and write a \texttt{main}-routine which performs a scaling analysis of the runtime needed to copy matrices with dimension $m \times m$ for $m\in\mathbb N$ and plot the runtime as a function of $m$.
	%
\end{itemize}
%

%
\section{Implementing fast matrix contractions \textbf{(10 Points)}}
%
For this exercise, checkout the code template \texttt{gemm\_wrapper}, which provides an elementary implementation of the fast \texttt{xgemm} operations.
%
Here, the \texttt{x} denotes the fundamental data type, i.e., \texttt{single}, \texttt{double}, \texttt{complex single}, or \texttt{complex double}.
%
The acronym \texttt{gemm} abbreviates \underline{GE}neral \underline{M}atrix-\underline{M}atrix multiplications and this convention carries over for other provided operations, for instance, \underline{GE}neral \underline{M}atrix-\underline{V}ector multiplications (checkout Intel's \href{https://www.intel.com/content/www/us/en/docs/onemkl/developer-guide-linux/2023-1/overview.html}{developer guide} for a rather complete documentation about the blas/lapack interface).
%
\begin{itemize}
	\item[(2.a)] \textbf{(5P)}
	%
	Proceed as in problem (1) and correct the code template.
	%
	Note how \texttt{la\_operations.h} now also contains a multiplication operations, which is compiled into an object file \texttt{la\_objects.o} that now implements a general matrix-matrix multiplication.
	%
	Write a test case, which tests the implemented \texttt{xgemm}-functionality and returns success (return code 0) or failure (return code 1) on exit, depending on whether a successful matrix-matrix multiplication has been performed.
	%
	Why is such a test case useful?
	%
	\item[(2.b)] \textbf{(5P)}
	%
	Write a trivial version of a matrix-matrix multiplication by implementing the calculations of the elements
	\begin{equation}
		C_{ij} = \sum_k A_{ik} B_{kj}
	\end{equation}
	of the result of a matrix-matrix product $\mathbf C = \mathbf A \cdot \mathbf B$, explicitely.
	%
	Here, we consider $\mathbf A, \mathbf B, \mathbf C \in \mathbb V^{m\times m}_\mathbb{R}$ for some $m\in \mathbb N$.
	%
	Perform a runtime analysis comparing your naive implementation with \texttt{xgemm} calls, investigating the dependency of the runtime on the matrix dimension $m$.
	%
	Interpret you result.
	%
\end{itemize}
%

%
\section{Syntactic sugar for matrix contractions \textbf{(5 Points)}}
%
For this exercise, checkout the code template \texttt{expr\_templates}, which provides an elementary implementation of expression templates to overload the multiplication operator.
%
In problem (2) we introduced an operator overload to the multiplication operator \texttt{*}, allowing for expression such as \texttt{C=A*B} in \Cpp.
%
However, that implementation also required an intermediate copy operation, which is necessary because \texttt{*} is a binary operator and the result of the matrix-matrix multiplication has to get through a temporary return value.
%
This unfortunate fact can be avoided by delayed evaluation, which in \Cpp{} can be implemented using \textit{expression templates}.
%
\begin{itemize}
	\item[(3.a)] \textbf{(3P)}
	%
	Proceed as in problem (1) and (2) and correct the code template.
	%
	In particular note how the binary matrix-matrix multiplication operator \texttt{*} is mapped to the unary assignment operator \texttt{=}, which assigns the result of the operation to an instance of \texttt{LAMatrix} without additional copy-operations.
	%
	\item[(3.b)] \textbf{(2P)}
	%
	Perform a runtime analysis and exctract the speed-up as a function of the matrix-dimension $m$ obtained, using the \texttt{xgemm}-impementations of the \texttt{*}-operator from problem (2) and (3).
	%
\end{itemize}
%
\batchmode  % This suppresses some verbose LaTeX output
\end{document}

