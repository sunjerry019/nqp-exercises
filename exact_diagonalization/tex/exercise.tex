% !TeX spellcheck = de_DE
\documentclass[a4paper, 12pt]{article}

\usepackage{bm, bbm, wrapfig, float, setspace, amsmath, amssymb, amsthm, url, graphicx, ngerman, transparent, enumerate, bbold, esint, polynom, hyperref, microtype, etoolbox, braket, cleveref, hyphenat}
\usepackage{centernot}

\usepackage{tikz}
\usepackage{epsdice}
\usepackage{listings}
\lstset{language=Python}
\tikzset{>=stealth}
\usetikzlibrary{decorations.markings, shapes, calc}

\renewcommand{\familydefault}{cmss}   % Generates sans serif fonts


\renewcommand{\l}{\left(}
\renewcommand{\r}{\right)}
\newcommand{\gs}{\text{gs}}
\renewcommand{\P}{\hat{P}}
\newcommand{\U}{\hat{U}}

% \newcommand{\bra}[1]{\langle#1|}
% \newcommand{\ket}[1]{|#1\rangle}
\newcommand{\bkt}[2]{\left\langle #1 |#2 \right\rangle}
\renewcommand{\ij}{{\langle \vec{i}, \vec{j} \rangle}}
\renewcommand{\H}{\hat{\mathcal{H}}}
\newcommand{\Ht}{\tilde{\mathcal{H}}}
\renewcommand{\c}{\hat{c}}
\renewcommand{\a}{\hat{a}}
\newcommand{\cd}{\hat{c}^\dagger}
\newcommand{\rh}{\hat{\rho}}
\newcommand{\rht}{\tilde{\rho}}
\newcommand{\ad}{\hat{a}^\dagger}
\newcommand{\bd}{\hat{b}^\dagger}
\newcommand{\ubd}{\hat{\uline{b}}^\dagger}
\newcommand{\ub}{\hat{\uline{b}}}
\renewcommand{\b}{\hat{b}}
\newcommand{\hd}{\hat{h}^\dagger}
\newcommand{\h}{\hat{h}}
\renewcommand{\d}{\hat{d}}
\newcommand{\n}{\hat{n}}
\newcommand{\D}{\hat{D}}
\newcommand{\Dd}{\hat{D}^\dagger~\hspace{-0.12cm}}

\newcommand{\G}{\hat{\Gamma}}
\newcommand{\Gd}{\hat{\Gamma}^\dagger}
\newcommand{\F}{\hat{F}}
\newcommand{\Fd}{\hat{F}^\dagger}
\newcommand{\hc}{\text{h.c.}}
\newcommand{\MF}{\text{MF}}
\newcommand{\BEC}{\text{BEC}}
\newcommand{\RG}{\text{RG}}
\newcommand{\psd}{\hat{\psi}^\dagger}
\newcommand{\ps}{\hat{\psi}}
\newcommand{\I}{\text{I}}
\newcommand{\p}{\text{p}}
\newcommand{\f}{\text{F}}
\newcommand{\s}{\text{S}}
\renewcommand{\sf}{\text{MIX}}
\renewcommand{\O}{\hat{\mathcal{O}}}
\newcommand{\W}{\hat{W}}
\newcommand{\Ud}{\hat{U}^\dagger}
\newcommand{\HMF}{\mathscr{H}_{\text{MF}}}
\newcommand{\ph}{\text{ph}}
\newcommand{\IB}{\text{IB}}
\newcommand{\B}{\text{B}}
\newcommand{\eff}{\text{eff}}
\newcommand{\tr}{\text{tr}}
\newcommand{\tdiff}{\,\mathrm{d}}



\normalsize
\setlength{\hoffset}{-1.5cm}
\addtolength{\textwidth}{3cm}
\setlength{\voffset}{-2.2cm}
\addtolength{\textheight}{3.5cm}
\addtolength{\footskip}{0.2cm}

%\input{macros.tex}
\newtheorem{aufgabe}{Problem}
\newtheorem{bspaufgabe}{Beispielaufgabe}

\usepackage{titlesec}

\titleformat{\section}[block]
{\normalfont}{\textbf{Problem \thesection}}{1em}{}

\DeclareMathOperator{\dd}{d}
\newcommand{\eul}{\text{e}}

% imaginary unit 
\newcommand{\imag}{\textsl{i}}
%
% with or without solution?
%
\makeatletter
\@ifundefined{solfalse}{
\newif\ifsol
\soltrue
}
\makeatother
\newcommand{\solA}{\vskip0.5em \color{brown}\noindent\textbf{Solution:\\}}
\newcommand{\sol}[1]{\vskip0.5em \color{brown}\noindent\textbf{Solution:}\\#1}

\usepackage{relsize}
\newcommand\Cpp{C\nolinebreak[4]\hspace{-.05em}\raisebox{.4ex}{\relsize{-3}{\textbf{++}}}}

\begin{document}
	   
% ********** First Box **************
\vspace*{-13mm}
\noindent\fbox{
	\parbox[b][19mm][b]{16mm}{%\parbox[POS][H�HE][POS-INNEN]{BREITE}{INHALT}
		\center\includegraphics[trim=1.6mm  0 0 0, width=18mm]{../../images/lmu-bildmarke.png}
	}
}
\hfill
% ********** Second Box **************
% Diese Schrift sollte so gross sein, dass die Oberkante des Ms von
% Max mit der Oberkante von LMU abschliesst
\fbox{\parbox[b][19mm][b]{19mm}{\includegraphics[width=19mm,height=19mm]{../../images/lmu-wortmarke.png}}}%
\hfill\hfill
% ********** Third Box **************
\fbox{\parbox[b][19mm][b]{84mm}{\fontsize{9}{12}\sc Faculty of Physics, Summer Term 2023 \\  Numerical Quantum Physics\\ 
		Lecturer: Dr. S. Paeckel \\ Assistant Lecturer: Z. Xie, B. Schneider
		\vspace{0.1mm}}}
\hfill
% ********** Fourth Box **************
\fbox{
	\parbox[b][19mm][b]{25mm}{\hspace*{3mm}\transparent{0.52}{\vspace{-1mm}\includegraphics[trim=0 6mm 16mm 20mm, clip, height=21mm]{../../images/lmu-siegel.png}}\transparent{1}}
}
\begin{center}
	\small \url{https://www2.physik.uni-muenchen.de/lehre/vorlesungen/sose_23/nqp/}
\end{center}
%
\vspace{8mm}
%
\centerline{\Large\textbf{Sheet~2:~Exact Diagonalization}}
%
\vspace{3mm}
%
\normalsize\centerline{Released:~05/05/23;~Submit until:~05/19/23 (\textbf{20 Points})}
%
%\vspace{0mm}
%
%
\vspace{6mm}
%
On this sheet we will set up a small exact diagonalization code to solve a fundamental problem of quantum mechanics: the one-dimensional transverse-field Ising model.
%
Solving the exercises, we will try to follow the paradigm of \textit{test-driven development}, which is a standard paradigm when developing complex code.
%
There are many neat tools for various languages.
%
If you are working in \texttt{Python}, you may find it helpful to have a closer look at the \texttt{Pytest} framework, which is already available on ASC cluster (after sourcing \texttt{init\_modules.sh}).
%
If your are working in \texttt{C++}, you can use the header file \texttt{catch.hpp} in the folder \texttt{exact\_diagonalization} of the exercise-git to set up unit tests (as for instance explained \href{https://medium.com/dsckiit/a-guide-to-using-catch2-for-unit-testing-in-c-f0f5450d05fb}{here}).
%

%
\section{A tensor product Hilbert space \textbf{(8 Points)}}
%
We are considering a chain of $L\in\mathbb N$ spins, each of which being described by a two-dimensional degree of freedom $\ket{s_j} \in \mathbb C^2$ where $j\in\left\{0,\ldots,L-1 \right\}$ labels the lattice site.
%
The transverse-field Ising model with periodic boundary conditions in one dimension is now defined by the Hamiltonian
\begin{equation}
	\hat H = -\sum_{j=0}^{L-1} \hat \sigma^z_j \hat \sigma^z_{j+1} - h \sum_{j=0}^{L-1} \hat \sigma^x_j \;, \label{eq:tfim}
\end{equation}
where $h\in\mathbb R$ is the transverse magnetic field.
%
Here, the $\hat \sigma^\alpha_j$ are Pauli operators acting on the $j$th spin of the many-body Hilbert space $\mathcal H = \bigotimes_{j=0}^{L-1} \mathbb C^2$, fulfilling the commutation relations
\begin{equation}
	 \left[ \hat \sigma^\alpha_i, \hat \sigma^\beta_j \right] = 2\delta_{ij}\epsilon^{\alpha\beta\gamma}\hat\sigma^\gamma_j \;,
\end{equation}
where $\alpha,\beta,\gamma\in \left\{x,y,z \right\}$ and $\epsilon^{\alpha\beta\gamma}$ is the Levi-Civita symbol.
%

%
Implement a class that represents the Hilbert space $\mathcal H$ and operators acting on it, for a given number of lattice sites $L$.
%
It should at least provide the following functionality:
\begin{itemize}
	\item Generation of a random state, a ferromagnetic state along the $z$-direction and a ferromagnetic state along the $x$-direction, represented as vectors $\vec x \in \mathbb C^{2^L}$.
	\item Generation of operators $\hat \sigma^\alpha_j$ as well as the identity, acting on $\mathcal{H}$, represented as matrices $\mathbf M \in \mathbb C^{2^L \times 2^L}$.
	\item Action of an operator $\hat O$ on a state $\ket{\psi} \in \mathcal H$.
	\item Calculation of the standard scalar product $\braket{\cdot | \cdot}$ on $\mathcal H$.
\end{itemize}
%
For each functionality, write a proper testcase which reasonably validates your implementation.
%

%
\section{Determining the ground-state phase diagram\textbf{(12 Points)}}
%
Now that we have set the stage, we can start to study~\cref{eq:tfim}, numerically.
%
\begin{itemize}
	\item[(2.a)] \textbf{(3P)}
	%
	Implement a function which generates a matrix representation of $\hat H$ as a function of the transverse field.
	%
	Write at least two testcases, which reasonably validate your implementation.
	%
	\item[(2.b)] \textbf{(3P)}
	%
	Calculate the ground-state energy density $e(h) = \braket{\hat H}/L$ as a function of the transverse field $h$ by determining the ground state energy of $\hat H$.
	%
	Also, evaluate the average ground-state magnetization as a function of the transverse field
	\begin{equation}
		m(h) = \frac{1}{2L} \sum_{j=0}^{L-1} \braket{\hat \sigma^z_j} \;.
	\end{equation}
	%
	\item[(2.c)] \textbf{(4P)}
	%
	Repeat your calculations for different number of lattice sites and determine the critical field $h_\mathrm{c}(L)$ at which the model undergoes a quantum phase transition.
	%
	Perform a finite-size scaling of $h_\mathrm{c}(L)$ to estimate $h_\mathrm{c}(L\rightarrow \infty)$, i.e., the critical field in the thermodynamic limit.
	%
	How does your result compare to the literature?
	%
	\item[(2.d)] \textbf{(2P)}
	%
	Show numerically that in the limits $h\rightarrow 0$ and $h\rightarrow \infty$, the ground states $\ket{\psi_0(h)}$ of~\cref{eq:tfim} approach the mean-field solutions
	\begin{equation}
		\ket{\psi_0(h)}
		=
		\begin{cases}
			\ket{\uparrow}\otimes\ket{\uparrow} \cdots \text{ or } \ket{\downarrow}\otimes\ket{\downarrow} \cdots\;, &\text{if $h\rightarrow 0$,} \\
			\ket{\rightarrow}\otimes\ket{\rightarrow} \cdots\;, &\text{if $h\rightarrow \infty$.} \\
		\end{cases}
	\end{equation}
	%
\end{itemize}
%
\batchmode  % This suppresses some verbose LaTeX output
\end{document}

